\section{面向切面编程(AOP)}\label{ux9762ux5411ux5207ux9762ux7f16ux7a0baop}

\subsection{AOP 统一记录 HTTP
请求日志}\label{aop-ux7edfux4e00ux8bb0ux5f55-http-ux8bf7ux6c42ux65e5ux5fd7}

\begin{itemize}
\tightlist
\item
  实现思路同样适用于非 HTTP 请求类型日志记录.
\item
  本文需求是: 通过日志记录 Controller 中的请求.
\item
  本文不对日志相关的配置作说明.
\item
  完整示例可以直接看参考.
\end{itemize}

\subsubsection{环境}\label{ux73afux5883}

\begin{itemize}
\tightlist
\item
  apache-tomcat-8.5.11
\item
  jdk1.8.0\_121 (1.7 也可以)
\end{itemize}

\subsubsection{配置}\label{ux914dux7f6e}

maven \lstinline!pom.xml! 配置:

\begin{lstlisting}
<dependency>
    <groupId>org.aspectj</groupId>
    <artifactId>aspectjrt</artifactId>
    <version>1.8.4</version>
</dependency>
<dependency>
    <groupId>org.aspectj</groupId>
    <artifactId>aspectjweaver</artifactId>
    <version>1.8.4</version>
</dependency>
<dependency>
    <groupId>cglib</groupId>
    <artifactId>cglib</artifactId>
    <version>2.2</version>
</dependency>
\end{lstlisting}

\lstinline!dispatcher-servlet.xml! 配置:

\begin{lstlisting}
<context:component-scan base-package="com.xx.xxxx" />
<aop:aspectj-autoproxy />
\end{lstlisting}

\subsubsection{AOP
日志记录实现}\label{aop-ux65e5ux5fd7ux8bb0ux5f55ux5b9eux73b0}

\begin{lstlisting}
import org.apache.log4j.Logger;
import org.aspectj.lang.JoinPoint;
import org.aspectj.lang.annotation.Aspect;
import org.aspectj.lang.annotation.Pointcut;
import org.aspectj.lang.annotation.AfterReturning;
import org.aspectj.lang.annotation.Before;
import org.springframework.core.annotation.Order;
import org.springframework.stereotype.Component;
import org.springframework.web.context.request.RequestContextHolder;
import org.springframework.web.context.request.ServletRequestAttributes;

import javax.servlet.http.HttpServletRequest;
import java.util.Arrays;

/**
* Order(3) 制定 Aspect 处理顺序, 数值越小, 优先级越高
*/
@Aspect()
@Order(3)
@Component()
public class HttpLogAspect {

    private Logger logger = Logger.getLogger(getClass());
    private ThreadLocal<Long> startTime = new ThreadLocal<Long>(); // 记录请求与响应花费的时间

    @Pointcut("within(@org.springframework.stereotype.Controller *)")
    public void controller() {}

    @Pointcut("execution(* *.*(..))")
    protected void allMethod() {}

    /**
    * 执行前
    * 记录 HTTP 请求详细
    * @param joinPoint joinPoint
    */
    @Before("controller() && allMethod()")
    public void logBefore(JoinPoint joinPoint) {
        // 开始计时
        startTime.set(System.currentTimeMillis());

        logger.info("** START HTTP REQUEST **");

        ServletRequestAttributes attributes = (ServletRequestAttributes) RequestContextHolder.getRequestAttributes();
        HttpServletRequest request = attributes.getRequest();

        // 记录类名及方法名
        logger.info("HTTP_CLASS_METHOD : " + joinPoint.getSignature().getDeclaringTypeName() + "."
            + joinPoint.getSignature().getName());
        // 记录请求参数
        logger.info("HTTP_ARGUMENTS :  " + Arrays.toString(joinPoint.getArgs()));

        if (null != request) {
            // 记录请求地址
            logger.info("HTTP_REQUEST_URL : " + request.getRequestURL().toString());
            // 记录请求方法
            logger.info("HTTP_METHOD : " + request.getMethod());
            // 记录请求 IP
            logger.info("HTTP_REQUEST_IP : " + request.getRemoteAddr());
        }
    }

    /**
    * 执行后
    * 请求结束, 记录返回内容
    * @param result 响应内容
    */
    @AfterReturning(pointcut = "controller() && allMethod()", returning = "result")
    public void logAfterReturning(Object result) {
        logger.info("HTTP_RESPONSE : " + result);
        // 结束计时
        logger.info("HTTP_SPEND_TIME : " + (System.currentTimeMillis() - startTime.get()) + " ms");
        logger.info("** END HTTP REQUEST **");
    }

}
\end{lstlisting}

\section{maven 使用相关}\label{maven-ux4f7fux7528ux76f8ux5173}

\subsection{不同环境(开发,上线)配置切换}\label{ux4e0dux540cux73afux5883ux5f00ux53d1ux4e0aux7ebfux914dux7f6eux5207ux6362}

在编译时使用 maven 命令参数打包不同环境下的配置文件, 比如
\lstinline!src/main/resources/prod! 和
\lstinline!src/main/resources/dev!
文件夹下分别是线上环境和开发环境的配置文件. maven \lstinline!pom.xml!
配置文件部分配置如下.

\begin{lstlisting}[language=XML]
<build>
    <resources>
        <resource>
            <directory>src/main/resources</directory>
            <!-- 资源根目录排除各环境的配置,使用单独的资源目录来指定 -->
            <excludes>
                <exclude>prod/*</exclude>
                <exclude>dev/*</exclude>
            </excludes>
        </resource>
        <resource>
            <directory>src/main/resources/${profiles.active}</directory>
        </resource>
    </resources>
</build>
<profiles>
    <profile>
        <!-- 开发环境 -->
        <id>dev</id>
        <properties>
            <profiles.active>dev</profiles.active>
        </properties>
        <activation>
            <activeByDefault>true</activeByDefault>
        </activation>
    </profile>
    <profile>
        <!-- 生产环境 -->
        <id>prod</id>
        <properties>
            <profiles.active>prod</profiles.active>
        </properties>
    </profile>
</profiles>
\end{lstlisting}

打包时通过 \lstinline!mvn clean package -P prod! 实现线上环境配置打包,
\lstinline!mvn clean package -P dev! 实现开发环境配置打包.

\subsection{本地 JAR
文件引入}\label{ux672cux5730-jar-ux6587ux4ef6ux5f15ux5165}

\textbf{本地 JAR 文件加入到本地 maven 库}

\begin{lstlisting}[language=XML]
<!--build>plugins 下添加-->
<!--安装本地 jar 包-->
<!-- https://mvnrepository.com/artifact/org.apache.maven.plugins/maven-install-plugin -->
<plugin>
    <groupId>org.apache.maven.plugins</groupId>
    <artifactId>maven-install-plugin</artifactId>
    <version>2.5.2</version>
    <configuration>
        <groupId>com.mycompany</groupId>
        <artifactId>myproject</artifactId>
        <version>1.0</version>
        <packaging>jar</packaging>
        <generatePom>true</generatePom>
        <file>${basedir}/src/main/webapp/WEB-INF/lib/myjar.jar</file>
    </configuration>
    <executions>
        <execution>
            <id>install-jar-lib</id>
            <goals>
                <goal>install-file</goal>
            </goals>
            <phase>validate</phase>
        </execution>
    </executions>
</plugin>
\end{lstlisting}

\textbf{依赖添加}

\begin{lstlisting}[language=XML]
<!--dependencies 下添加-->
<dependency>
    <groupId>com.mycompany</groupId>
    <artifactId>myproject</artifactId>
    <version>1.0</version>
</dependency>
\end{lstlisting}

\textbf{打包前先执行} \lstinline!mvn install:install-file! 或者
\lstinline!mvn validate!. 或者一行命令执行
\lstinline!mvn validate & mvn clean package -P dev!.

\begin{itemize}
\tightlist
\item
  \href{https://www.cnblogs.com/xguo/archive/2013/06/04/3117894.html}{Maven安装jar文件到本地仓库}
\item
  \href{https://stackoverflow.com/questions/26618192/install-local-jar-dependency-as-part-of-the-lifecycle-before-maven-attempts-to}{Install
  local jar dependency as part of the lifecycle, before Maven attempts
  to resolve it}
\end{itemize}
